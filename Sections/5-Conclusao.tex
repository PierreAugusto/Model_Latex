\chapter{\chapternameconclusions}\label{chap:conclusions}

Este trabalho apresentou uma busca sistemática por galáxias anãs ultra-compactas (UCDs) no aglomerado de Fornax, integrando análises fotométricas (Capítulo \ref{cap:database}), técnicas de aprendizado de máquina (Capítulo \ref{cap:analise}) e observações espectroscópicas obtidas com o telescópio Gemini Sul (Capítulos \ref{chap:spectra} e \ref{chap:spectra_emission}). A combinação dessas abordagens permitiu a seleção de 14 novas candidatas principais a UCDs, cujas propriedades morfológicas e fotométricas foram analisadas detalhadamente.

A abordagem baseada em aprendizado de máquina se mostrou eficaz na tarefa de separar objetos compactos e extensos, especialmente ao utilizar conjuntamente magnitudes e cores. O classificador Random Forest, em particular, obteve métricas robustas, com AUC-ROC de 0.97 e MCC de 0.84. A análise das candidatas revelou que elas se distribuem nas mesmas regiões de espaço de parâmetros que as UCDs conhecidas, como evidenciado pelos diagramas envolvendo o brilho superficial médio, \textit{MU\_MAX} e \textit{KRON\_RADIUS}.

Outro resultado relevante está relacionado à distribuição espacial dessas candidatas. Observamos que diversas delas estão localizadas fora do raio virial do aglomerado de Fornax, o que sugere que a população de UCDs pode se estender além das regiões centrais normalmente exploradas. Essa distribuição pode estar relacionada à história dinâmica do aglomerado e levanta hipóteses sobre processos de formação em ambientes menos densos.

Na etapa espectroscópica, foram analisados 18 objetos, dos quais 14 possuíam espectros obtidos com o GMOS-S no Gemini Sul, provenientes de um projeto anterior. Essa análise foi realizada no início do presente projeto de mestrado. Embora nenhum desses objetos tenha sido confirmado como membro do aglomerado de Fornax, sendo classificados como estrelas do primeiro plano, galáxias de fundo ou quasares, o procedimento ilustra a importância da espectroscopia para a validação das candidatas.

Com base nas metodologias desenvolvidas neste trabalho, foi possível identificar um novo conjunto de 14 candidatas a UCDs, descritas na Seção \ref{sec:selecao_candidatas} e analisadas na Seção \ref{sec:analise_candidatas}. Essas candidatas apresentam propriedades consistentes com UCDs conhecidas e mostram-se promissoras para estudos futuros. No entanto, essas novas candidatas ainda não foram observadas espectroscopicamente, sendo necessário solicitar tempo de observação para confirmar sua natureza em análises posteriores.

Além disso, foi explorada a presença de objetos compactos com sinais de emissão, inicialmente identificados com base em um pico no filtro $J0660$. Embora os objetos analisados espectroscopicamente nesta etapa apresentem redshifts elevados ($\sim 0.3$), e, portanto, estejam fora do aglomerado, sua seleção ilustra que a metodologia proposta também é eficaz para identificar galáxias compactas emissoras (como possíveis BCDs). A inclusão de um corte em cor envolvendo o $J0660$ pode ser útil para buscas mais refinadas desses objetos.

Apesar das limitações, como a amostra espectroscópica reduzida e a possibilidade de contaminações residuais na seleção, os resultados obtidos validam a estratégia proposta nesse trabalho. A metodologia adotada, especialmente com o uso de aprendizado de máquina junto dos critérios fotométricos e morfológicos, se mostrou eficaz na identificação de novas candidatas a UCDs e pode ser replicada em outras regiões e levantamentos.