\thispagestyle{empty}
\chapter*{Resumo}
Este trabalho apresenta uma busca sistemática por galáxias anãs compactas no aglomerado de Fornax, combinando fotometria em doze bandas do S-PLUS com classificação baseada em aprendizado de máquina. A distinção entre UCDs e outros objetos compactos, como estrelas ou aglomerados globulares, é desafiadora. Para abordar essa dificuldade, empregamos um método supervisionado que classifica os objetos em duas categorias com base exclusivamente na fotometria: compactos e extensos. Essa abordagem permite identificar objetos compactos com características fotométricas semelhantes às de galáxias, facilitando a seleção de candidatas a UCDs. Esta análise foi realizada em uma região do céu cobrindo cinco vezes o raio virial do aglomerado.

Testamos alguns classificadores e um algoritmo Random Forest, otimizado via validação cruzada, alcançou uma área sob a curva ROC (AUC-ROC) de 0,97 e um coeficiente de correlação de Matthews de 0,84. Com base nas probabilidades geradas pelo classificador, foram aplicados cortes morfológicos e restrições de \textit{photo-z}, resultando na identificação de 14 novos candidatos a galáxias anãs compactas em regiões que se estendem além do raio virial estimado de $\sim700$ kpc de Fornax. Se confirmados espectroscopicamente, este resultado indicaria a presença desses sistemas em ambientes externos ao aglomerado, diferentes daqueles onde as UCDs são normalmente encontradas.

Além disso, avaliamos o uso do filtro estreito $J0660$ para identificar características de emissão que pudessem indicar galáxias compactas em formação estelar, como as anãs compactas azuis (BCDs). Embora as observações iniciais realizadas com o GMOS-S no telescópio Gemini tenham revelado redshifts de $z \sim 0.3$, esse experimento valida a inclusão de cortes em cores de banda estreita em levantamentos futuros de emissores compactos.

Nossa abordagem demonstra que metodologias automatizadas podem ampliar significativamente o censo de sistemas estelares compactos em ambientes densos.

\textbf{Palavras-chave}: Galáxias anãs ultra-compactas, galáxias compactas azuis, aglomerado de Fornax, aprendizado de máquina, S-PLUS.


\chapter*{Abstract}
This work presents a systematic search for compact dwarf galaxies in the Fornax cluster, combining twelve-band photometry from S-PLUS with machine learning-based classification. The distinction between UCDs and other compact objects, such as stars or globular clusters, is challenging. To address this difficulty, we employed a supervised method that classifies objects into two categories based exclusively on photometry: compact and extended. This approach enables the identification of compact objects with photometric characteristics similar to galaxies, facilitating the selection of UCD candidates. This analysis was conducted in a sky region covering five times the virial radius of the cluster.

We tested several classifiers, and a Random Forest algorithm, optimized via cross-validation, achieved an area under the ROC curve (AUC-ROC) of 0.97 and a Matthews correlation coefficient of 0.84. Based on the probabilities generated by the classifier, morphological cuts and \textit{photo-z} constraints were applied, resulting in the identification of 14 new candidates for compact dwarf galaxies in regions extending beyond the estimated virial radius of $\sim700$ kpc of Fornax. If spectroscopically confirmed, this result would indicate the presence of these systems in environments external to the cluster, distinct from those where UCDs are typically found.

Additionally, we evaluated the use of the narrow $J0660$ filter to identify emission features that could indicate compact galaxies undergoing star formation, such as blue compact dwarfs (BCDs). Although initial observations conducted with GMOS-S on the Gemini telescope revealed redshifts of $z \sim 0.3$, this experiment validates the inclusion of narrow-band color cuts in future surveys of compact emitters.

Our approach demonstrates that automated methodologies can significantly expand the census of compact stellar systems in dense environments.

\textbf{Keywords}: Ultra-compact dwarf galaxies, blue compact galaxies, Fornax cluster, machine learning, S-PLUS.

\thispagestyle{empty}