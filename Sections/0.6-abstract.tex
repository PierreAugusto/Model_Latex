\thispagestyle{empty}
\chapter*{Resumo}
Este trabalho investiga as \ac{UCD} no aglomerado de Fornax, com o objetivo de aprofundar a compreensão de suas propriedades físicas, mecanismos de formação e evolução em ambientes densos. Motivada pela natureza híbrida desses objetos – que exibem características intermediárias entre \ac{GC} e núcleos de galáxias nucleadas, a pesquisa propõe a integração de dados fotométricos do S-PLUS com observações espectroscópicas realizadas pelo telescópio Gemini Sul.

A metodologia desenvolvida se baseou na aplicação de técnicas de aprendizado de máquina para a identificação e classificação de candidatas a \ac{UCD}, permitindo a distinção entre objetos estelares e galácticos. Essa abordagem possibilitou a busca de objetos compactos com características semelhantes a galáxias, nas regiões que se estendem além do raio do virial do aglomerado. A análise combinou a estimação de redshifts fotométricos e espectroscópicos para a seleção de alvos, bem como a avaliação de parâmetros morfológicos e estruturais para a classificação de UCDs genuínas.

Outro aspecto importante foi a aplicação do filtro $J0660$, que permitiu a detecção de sinais de emissão em alguns candidatos – evidência que pode indicar processos de formação ativa ou a presença de núcleos ativos. Esses achados contribuem para a compreensão dos processos de formação das UCDs, reforçando a importância do uso de metodologias automatizadas na astronomia moderna e abrindo possibilidades para futuras investigações sobre a evolução de sistemas compactos.

Os capítulos subsequentes deste trabalho detalham a análise dos dados coletados, iniciando com a descrição dos métodos de observação e coleta de dados no Capítulo \ref{cap:database}. O Capítulo \ref{cap:analise} explora as técnicas de processamento de dados, incluindo a aplicação de algoritmos de aprendizado de máquina para a identificação de UCDs, além das análises fotométricas. O Apêndice \ref{chap:spectra} e o Capítulo \ref{chap:spectra_emission} discutem os dados espectroscópicos e suas análises. Por fim, o Capítulo \ref{chap:conclusions} apresenta as conclusões obtidas.


\textbf{Palavras-chave}: Galáxias anãs ultra-compactas, galáxias compactas azuis, aglomerado de Fornax, aprendizado de máquina, S-PLUS.

\chapter*{Abstract}
This work investigates ultra-compact dwarf galaxies (UCDs) in the Fornax cluster, aiming to deepen the understanding of their physical properties, formation mechanisms, and evolution in dense environments. Motivated by the hybrid nature of these objects – which exhibit intermediate characteristics between globular clusters and nuclei of nucleated galaxies, the research proposes the integration of photometric data from S-PLUS with spectroscopic observations conducted by the Gemini South telescope.

The developed methodology is based on the application of machine learning techniques for the identification and classification of UCD candidates, allowing the distinction between stellar and galactic objects. This approach enabled the search for compact objects with galaxy-like characteristics in regions extending beyond the virial radius of the cluster. The analysis combined the estimation of photometric and spectroscopic redshifts for target selection, as well as the evaluation of morphological and structural parameters for the classification of genuine UCDs.

Another important aspect was the application of the $J0660$ filter, which allowed the detection of emission signals in some candidates – evidence that may indicate active formation processes or the presence of active nuclei. These findings contribute to the understanding of UCD formation processes, reinforcing the importance of using automated methodologies in modern astronomy and opening possibilities for future investigations on the evolution of compact systems.

The subsequent chapters of this work detail the analysis of the collected data, starting with the description of the observation and data collection methods in Chapter \ref{cap:database}. Chapter \ref{cap:analise} addresses the data processing techniques, including the application of machine learning algorithms for the identification of UCDs, as well as photometric analyses. Appendix \ref{chap:spectra} and Chapter \ref{chap:spectra_emission} discuss the spectroscopic data and their analyses. Finally, Chapter \ref{chap:conclusions} presents the conclusions obtained.

\textbf{Keywords}: Ultra-compact dwarf galaxies, blue compact galaxies, Fornax cluster, machine learning, S-PLUS.

\thispagestyle{empty}