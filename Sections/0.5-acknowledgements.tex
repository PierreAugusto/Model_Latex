\thispagestyle{empty}%\addtocounter{page}{-1}
\pagenumbering{gobble}
% \null
% \vfill
% \hfill%
% \begin{minipage}{.50\linewidth}
%   \textit{Dedicatória.}
% \end{minipage}

\chapter*{Agradecimentos}
Ao final dessa primeira parte da jornada, ainda parece tudo tão imprevisível e incerto quanto quando eu comecei, mas, da mesma maneira, é tudo tão empolgante quanto antes. Se eu pudesse voltar ao passado e contar para o meu eu de 10 anos que conseguimos nos tornar astrônomos, tenho certeza de que ele ficaria muito feliz. E é assim que me sinto agora também.

Gostaria de agradecer à minha família, que, desde a primeira vez, quando pequeno, mostrei interesse pelos mistérios do universo e pela ciência, sempre me apoiou e me incentivou a correr atrás de realizar meus sonhos. Muito obrigado a todos vocês pelo apoio emocional e financeiro ao longo de todos esses anos.

Meus agradecimentos ao meu orientador, Laerte Sodré, por sua paciência e ajuda na minha evolução ao longo do projeto. Sua orientação foi fundamental para o desenvolvimento deste trabalho, tenho muita gratidão e respeito por você. Um obrigado também para a professora Cláudia, que me acompanhou desde o início da minha jornada na astronomia, me orientando na iniciação científica ao longo de muitos anos, sempre estando presente e me incentivando a progredir como pesquisador. Obrigado por toda a ajuda e suporte.

Um agradecimento gigante por todas as pessoas que passaram pela minha vida durante o tempo de graduação e mestrado. Obrigado aos meus amigos do começo da graduação, da nossa tropinha, por todas as confusões, risadas, desesperos que passamos juntos durante a pandemia, pelas aventuras e desafios que enfrentamos juntos dentro e fora da faculdade, o apoio e amizade de vocês foram fundamentais para eu conseguir chegar até aqui.

Obrigado à Cherateria e a todas as pessoas que conheci e me envolvi por causa da bateria. Esse coletivo se mostrou mais do que um hobby, mas também se tornou meu ponto de fuga do dia a dia durante a faculdade. Mais do que apenas a bateria e a música, levarei para sempre as amizades que fiz por causa dela.

Gostaria de agradecer ao escotismo e às pessoas que passaram pela minha vida por causa desse movimento. Eu sou quem sou por causa de todas as experiências que tive e de tudo que aprendi com o escotismo. Levarei ensinamento para minha vida toda, estando "Sempre Alerta" para cumprir minha promessa. Afinal, "Uma vez escoteiro, sempre escoteiro". Obrigado a todos os meus amigos e irmãos escoteiros.

Por fim, agradeço à CAPES pelo apoio financeiro deste projeto de mestrado, sem o qual não teria conseguido realizar este trabalho.



\cleardoublepage % Ensure "Frase" starts on an odd page
\null
\vfill
\hfill%
\begin{minipage}{.65\linewidth}
    \textit{Mesmo nas noites mais escuras, as estrelas ainda brilham.}

    \vspace{2cm} % Add vertical space for better separation

    \hfill
    \textit{"Deixe o mundo um pouco melhor do que encontrou."}
    
    \hfill
    \textit{- Robert Baden-Powell}

\end{minipage}


% \begin{flushright}
% \textbf{O Universo}\\
% \textit{Ferreira Gullar}
% \end{flushright}

% \begin{quote}
% O que vi do universo até hoje foi pouco, mas, se penso em quanto meço, posso dizer que foi muito.\\
% Sei, de ler, que o universo é de tais dimensões que a própria luz só o atravessa depois de bilhões e bilhões de anos, e que nele há multidões de galáxias e sóis que talvez já morreram, antes de chegar a sua luz até nós.\\
% Deste modo, é correto dizer que o céu que ora espio é passado e que até pode ser que o universo que veja já tenha se acabado.\\
% Mas, de fato, não vejo a não ser nas revistas de astronomia: o lampejo espantoso de infinitas constelações a brilhar num abismo espectral e difuso de gases e poeira estelar que me deixa confuso.\\
% E assim, assustado e mudo, bem menor que um ínfimo grão de poeira, contudo, sou capaz de apreender, no meu íntimo, essas incontáveis galáxias, esses espaços sem fim, essa treva e explosões de lava. Como tudo isso cabe em mim?\\
% O fato é que qualquer vasta nuvem prenhe de sóis já mortos ou futuros não possui consciência, esse obscuro fenômeno surgido aqui na Via Láctea, ou melhor, na Terra, e talvez somente nela, não se sabe por quê, mas que permite ao cosmos perceber-se a si mesmo, e ter olhos pra se ver.\\
% Olhos que são nossos, lentes minúsculas mas sensíveis que captam a luz das nebulosas vinda de espaço e tempo inconcebíveis.\\
% É o que dizem, pois tudo o que vejo é, à noite, apenas o brilhar de distantes luzes no escuro. São estrelas? planetas no sistema solar?\\
% Somos algo recente e raro no universo, como rara é também a própria luz dos sóis deste sol que nos aclara.\\
% Todo universo é treva. Inalcançável vastidão escura dentro da qual os sóis, as explosões de gás e luz são exceções.\\
% O universo em sua vastidão vazia é espaço e treva, é matéria fria em que não há o mínimo sinal de vida ou consciência; o que é mental nele, ao que se sabe, está em nós, no mínimo do mínimo do existente e o que também na treva luze é nossa voz inaudível no espantoso vão silente,\\
% Vi pouco do universo: afora a asa de luz e pó da Via Láctea, o que conheço são as manhãs que invadem minha casa.
% \end{quote}
