\chapter{\chapternamedatabase}\label{database}

Neste capítulo, apresentamos os dados utilizados neste trabalho, incluindo as fontes de dados fotométricos. O dados espectroscópicos serão apresentados e discutidos no capítulo 4. 

Primeiramente, descrevemos o levantamento S-PLUS, suas características e os dados obtidos. Em seguida, detalhamos os procedimentos de correção de extinção aplicados às magnitudes fotométricas. Apresentamos também como os dados foram seleiconados para a análise, incluindo os parâmetros utilizados para a fotometria e a seleção de galáxias, magnitudes limites e flags de qualidade.

\section{S-PLUS}
O Southern Photometric Local Universe Survey (S-PLUS), um levantamento do céu do hemisfério sul, localizado no Cerro Tololo Inter-American Observatory, Chile (CTIO), tem como objetivo observar uma área de 9300 graus quadrados do céu com o telescópio robótico de 80 cm T80-South \citep{oliveira2019splus}. Ele disponibiliza 12 filtros fotométricos (Figura \ref{splus_filters}), sendo 5 de banda larga ($u$, $g$, $r$, $z$) e 7 de banda estreita (J0378, J0395, J0410, J0430, J0515, J0660 e J0861).

Os dados utilizados neste trabalho foram obtidos a partir do quarto lançamento de dados do S-PLUS (DR4) \citep{herpich2024fourthsplusdatarelease}. O DR4 cobre uma área de 3022,7 graus quadrados, composta por 1629 campos, cada um deles com um campo de visão de aproximadamente 2 graus quadrados. A Figura \ref{footprint_iDR4} mostra a área coberta pelo S-PLUS DR4.

\begin{figure}[!ht]
    \begin{center}
    % \setcaptionmargin{1cm}
    \includegraphics[width=1.0 \columnwidth,angle=0]{footprint_iDR4.png}
    \caption[]{Área coberta pelo S-PLUS DR4. Campos em azul foram observados, enquanto campos em cinza não foram observados ainda \cite{splus_DR4_footprint}.}
    \label{footprint_iDR4}
    \end{center}
\end{figure}

Além dos 12 filtros fotométricos mencionados, temos opções dos tipos de magnitudes para cada uma dessas bandas. Entre os tipos de magnitudes disponíveis, destacam-se:

\begin{itemize}
    \item \textbf{ISO}: Captura o fluxo dentro de uma isofota, definida por um limite de brilho superficial constante.
    \item \textbf{PETRO}: Mede o fluxo utilizando a abertura de Petrosian, que captura uma fração constante da luz total da galáxia, minimizando a perda de luz nas regiões externas.
    \item \textbf{AUTO}: Abertura adaptativa que utiliza um algoritmo automático para ajustar a abertura ao tamanho e forma do objeto.
    \item \textbf{APER\_3}: Abertura fixa de 3 pixels de raio, adequada para medir o fluxo em regiões centrais.
    \item \textbf{APER\_6}: Abertura fixa de 6 pixels de raio, usada para capturar uma área maior do objeto, balanceando entre evitar contaminação de fontes próximas e capturar mais do fluxo total.
\end{itemize}

Para este trabalho, escolhemos utilizar as magnitudes baseadas na abertura APER\_6. A escolha foi motivada pela necessidade de uma medida consistente para os objetos mais compactos, como as galáxias ultracompactas (UCDs) que estamos buscando.

\begin{figure}[!ht]
    \begin{center}
    % \setcaptionmargin{1cm}
    \includegraphics[width=1.0 \columnwidth,angle=0]{splus_filters.png}
    \caption[]{Curvas de resposta dos 12 filtros fotométricos do S-PLUS \citep{splus_DR4_footprint}.}
    \label{splus_filters}
    \end{center}
\end{figure}

\subsection{Fotometria de Fornax}\label{sec:Fornax_data}
Os dados do aglomerado de Fornax foram obtidos a partir da mesma fonte que os dados do S-PLUS, o DR4. Porém, a partir da colaboração de \cite{castelli2024splusfornaxprojectsfp}, utilizamos os dados de \cite{haack2024splusfornaxprojectsfp}, onde foram realizadas novas execuções específicas do SExtractor no DR4, identificando o conjunto mais adequado de parâmetros para recuperar o máximo possível de galáxias de Fornax com fotometria confiável e evitando duplicações.

Os parâmetros da RUN 1 de \cite{haack2024splusfornaxprojectsfp} foram projetados para identificar galáxias fracas e objetos compactos próximos a galáxias mais brilhantes. Por outro lado, os parâmetros da RUN 2 são voltados para uma boa caracterização de galáxias brilhantes e extensas. Assim, para detectar objetos compactos como anãs ultra-compactas (UCDs) ou aglomerados globulares (GCs) perto das galáxias de Fornax, os parâmetros da RUN 1 são mais adequados. Já para caracterizar melhor os objetos mais brilhantes e extensos em Fornax, os parâmetros da RUN 2 são mais eficazes. No nosso caso, usamos os dados da RUN 1. Assim, todos as magnitudes e parâmetros dos SExtractor que usamos ao lono do trabalho são do RUN 1 de \cite{haack2024splusfornaxprojectsfp}.


\subsection{Correção da extinção}\label{sec:Coeficientes_ext}

As magnitudes utilizadas não estão corrigidas para a extinção interestelar. A poeira galáctica pode afetar significativamente as medições fotométricas, especialmente em áreas do céu com alta densidade de poeira. Para garantir que os dados fotométricos que utilizamos sejam precisos, é necessário aplicar uma correção de extinção.

Através do pacote Python \texttt{dustmaps} \citep{dustmapsGreen2018}, utilizamos o mapa de poeira CSFD \citep{chiang2023correctedsfdaccurategalactic}. O mapa CSFD é uma versão melhorada do mapa de Schlegel-Finkbeiner-Davis (SFD) (\citealt{Schlegel_1998} \& \citealt{Schlafly_2011}), que é comumente utilizado para estimar a extinção em diferentes direções do céu. A correção fornecida pelo CSFD ajuda a reduzir os efeitos da estrutura em larga escala e da contaminação do fundo infravermelho cósmico (CIB). Com esses efeitos removidos, o mapa CSFD fornece um valor mais preciso e confiável da extinção interestelar.

\textbf{Cálculo dos Coeficientes de Extinção}

Para converter as estimativas de extinção fornecidas pelo mapa de poeira em correções aplicáveis às magnitudes fotométricas, utilizamos a lei de extinção de \cite{cardelli1989dust}. Para cada uma das 12 magnitudes, calculamos os coeficientes de extinção com base nos comprimentos de onda efetivos das respectivas bandas.

\sloppy
O cálculo dos coeficientes de extinção foi realizado utilizando o pacote Python \texttt{extinction} de \cite{barbary2017extinction}, que implementa a lei de \cite{cardelli1989dust} e permite a aplicação direta das correções de extinção às magnitudes observadas.

\subsection{Cortes de qualidade na fotometria}\label{subsec:cuts}
Para garantir a qualidade dos dados fotométricos utilizados neste trabalho, aplicamos uma série de cortes de qualidade nas magnitudes. Esses cortes foram definidos com base em parâmetros específicos fornecidos pelo S-PLUS DR4.

Cortamos nas 12 magnitudes da abertura APER\_6 em magnitudes maiores do que 30, evitando objetos com baixa qualidade de medição e com erros fotométricos altos. Cortamos também em cada uma das mesmas bandas valores de \textit{flag0}$\leq$3, que são objetos com problemas de qualidade de medição.

Na seleção dos objetos utilizados nas próximas seções para a busca de galáxias ultracompactas, adotamos a banda \textit{g\_APER\_6} como referência principal. Aplicamos um corte inferior em \textit{g\_APER\_6}$\geq$13 para excluir objetos muito brilhantes e saturados, garantindo a qualidade das medições. Além disso, estabelecemos um corte superior em \textit{g\_APER\_6}$\leq$21, visando minimizar a contaminação por aglomerados globulares. Esses critérios de seleção vêem de \cite{Cantiello_2020}, que adotam um critério de separação de GCs de UCDs com $M_g=-10.5$, que corresponde $M_v\sim -11mag$ ($\sim10^7 M_\odot$) e uma magnitude aparente de $m_g=21$.

Dos dados do S-PLUS DR4, temos um total de 2900926 objetos. Após a aplicação dos cortes de qualidade, restaram 619630 objetos.

Das 12 magnitudes utilizadas no S-PLUS na abertura APER\_6 corrigidas, com os cortes descritos anteriormente, apresentamos na Figura \ref{fig:hist_distri_mags_data} as distribuições resultantes.

\begin{figure}[!ht]
    \begin{center}
    % \setcaptionmargin{1cm}
    \includegraphics[width=1.0 \columnwidth,angle=0]{hist_distri_mags_data.png}
    \caption[]{Distribuições de densidade das magnitudes corrigidas para extinção nas 12 bandas fotométricas do S-PLUS (APER\_6) após a aplicação dos cortes de qualidade descritos. As bandas incluem 5 de banda larga ($u$, $g$, $r$, $i$, $z$) e 7 de banda estreita (J0378, J0395, J0410, J0430, J0515, J0660, J0861). Os cortes garantem a seleção de objetos com medições fotométricas confiáveis para a análise.}
    \label{fig:hist_distri_mags_data}
    \end{center}
\end{figure}
